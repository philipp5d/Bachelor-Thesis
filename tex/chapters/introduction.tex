% !TEX root = ../Thesis.tex
\begin{document}
\documentclass[Thesis.tex]{subfiles}
\chapter{Introduction}
\label{ch:introduction}
%There are journals about case reports. -> It seems that information about individual cases is useful for doctors when reasoning about new patients. Indeed in the cognitive science literature it is documented, that humans often reason either case-based or prototype based. There is even a whole branch of AI dealing with learning to reason based on example cases, case-based reasoning.

%It seems like recommending cases similar to a current patient would be helpful for doctors for their clinical decision making. There are databases with a wealth of patient information, namely free text physician letters. However, search is limited to string matching approaches.
%-> A more elaborate automatic analysis of similarity between letters would be useful.

%For this task we obtained a small dataset from the university hospital Freiburg, on which we build a prototype of a recommender system for similar letters and explore other automatic information extraction procedures.

%--------------------------------------

\section{Motivation}

Publishing the case of a patient with a particularly interesting medical phenomenon in the form of a clinical case report has undergone a change in popularity in the medical community.
%The clinical case report, a scientific publication dealing with the specifics of one particularly interesting patient, has undergone a renewal of status in the medical community. 
The number of published case reports has been declining in standard journals. This has happened not only because case reports can hardly contribute to a good impact factor, but also because their scientific benefit has been questioned \citep{Mason2001}. However, several new journals dedicated only to case reports have emerged \citep{Kidd2007}\footnote{Additional Case Journals can be found at:\\Journal of Medical Cases: http://www.journalmc.org/index.php/JMC/index\\British Medical Journal Case Reports: http://casereports.bmj.com/site/about/index.xhtml\\American Journal of Medical Case Reports: http://www.sciepub.com/journal/AJMCR} and many people have argued for the value of case reports for research itself but also beyond \citep{Williams2003,Dib2008,Sandu2016}. Importantly case reports allow practitioners a more in-depth understanding of specific disease courses and provide educational material for students \citep{Nissen2014}. Although case reports lack the scientific validity of large empirical studies, it is apparent that people have strong intuitions for the usefulness of them. During our collaboration with practicing doctors, we also found that practitioners use the clinical records of similar patients to guide problem-solving for the current patient. Especially when faced with hard or unusual cases doctors seek similar patient information from the hospital database. While this only shows that doctors think that reviewing similar cases helps them in their work, we will argue that this can indeed improve their medical problem-solving. 

Cognitive scientists have discussed the usefulness of examples for reasoning processes for a long time and found that at least in some experimental settings reasoning processes are based on earlier presented examples \citep{Medin1978}. More recently these reasoning processes have also been studied in more realistic scenarios. \citet{Klein2008} reviews models for decision making under real world circumstances. According to him experts interpret a situation based on its resemblance to remembered situations. Once a sufficiently similar situation has been retrieved from memory, experts apply the solution from the remembered situation to the current one in a thought experiment. They evaluate whether or not this solution strategy will lead to success and adjust it, if necessary. In case no way to adequately adjust the solution can be found, another situation is retrieved from memory. This process is repeated until a sufficiently good solution is found. Presentation of similar cases should, therefore, aid doctors' decision making in an actual clinical setting. The medical domain has also been directly addressed by research in cognitive science. \citet{Elstein2002} have concluded that for medical problem-solving reasoning processes can be divided into two distinct categories. For cases perceived as easy doctors apply a kind of pattern recognition based on the examples they have encountered before and use solutions stored in memory. For harder cases, however, doctors need to rely on a more elaborate reasoning process. They have to consciously generate and eliminate hypotheses to be able to solve the problem. It is plausible that hypothesis generation, as well as hypothesis falsification, is also guided by the doctor's experience with earlier patients. From a more theoretical perspective, \citet{Kolodner1987} have specifically argued that ``[i]ndividual experiences act as exemplars upon which to base later decisions'' in medical problem-solving. Their research was partially driven by the desire to understand the way in which clinicians perform problem-solving but also by the goal of building artificial systems that can aid in this process. They argue that both humans and machines can learn from specific examples and use them to reason about new problems.

Around the idea that artificial systems might learn from examples has evolved a whole branch of Artificial Intelligence (AI), which is called ``Case-based reasoning''. This domain has been greatly influenced by psychological findings, some of them mentioned above. Researchers have successfully built systems used in real world applications, that reason from the examples provided  \citep{Aamodt1994}. Within this domain of AI, one of the greatest application areas is medicine. It seems that the medical area does not only offer straightforward usage of examples, but also has a need for automatic aids for problem-solving \citep{Begum2011}.

%Indeed in cognitive science a branch of research exists that is concerned with "exemplar" or "instance"-based reasoning. In exemplar theories it is assumed that humans categorize objects based on examples stored in memory \citep{Medin1978}. However, according to \citet{Elstein2002} medical decision making can be divided into two subcategories. In cases perceived as easily solvable doctors can use pattern recognition, which can be based on previous examples, to come to a decision. In hard cases though, experts fall back to conscious hypothesis generation and elimination. We want to deal mainly with hard cases, as experts will only then seek the help of a recommender system. In those hard cases experts use hypothesis testing and therefore similar examples can provide new ideas for hypothesis formulation or help remove irrelevant or wrong hypotheses. NDM, especially RPD model asserts that experts interpret a situation based on its resemblance to an old situation, but then think the current situation through with the old strategy and see whether it would work here as well. Similar cases can show a possible, plausible strategy path and then the doctor can think it through and take this strategy or reject it based on similarity between the cases.
%Even better use Kolodner Paper!

Given the practical, psychological and theoretical reflections above we believe that it would be helpful for practitioners to be able to review cases of similar patients. One particularly well-suited source for the retrieval of patient cases are databases of physician letters, as these letters provide concise summaries of the specifics of a patient that matter in practice. Search in these databases is, to our knowledge, usually limited to character matching procedures and therefore provides limited practical value for doctors. We, therefore, set out to build a prototypical recommender system on those physician letters to do automatic retrieval of only the relevant documents from a database.


\section{Dataset}

To get a dataset for exploratory work on the recommender system we collaborated with the University Hospital Freiburg. The hematology and oncology department has a database of approximately 190,000 German
physician letters in PDF form \citep{spadaro2012} (!correct! noch nicht zufrieden, wie das item in der bibliography erscheint. Ich weiß allerdings auch nicht wie es richtig wäre.). These physician letters are free text documents
written by the doctors to keep record of the patient's visit. They
usually include information about the patient's age, sex, diagnosed
diseases, therapy history, current complaints, many more medical details
like blood counts, but also personal information like names and birth dates. One of these letters is shown in the appendix in figures \ref{fig:letter_first_page} and \ref{fig:letter_second_page}.
The letters generally follow a rough structure. Almost all of them include a letterhead (a greeting and introduction), a diagnosis (summarizing diagnosed diseases bullet point like), a therapy history (listing the past therapies with dates) and an anamnesis (free text about current complaints etc.) section, separated into individual paragraphs. In principal though, doctors are free to document this information in the way they please. The database does not, however, contain the information of 190,000 unique patients. For many patients several letters are included, as a new visit will often result in an updated letter, that is added to the database. We refer to these letters as ``follow-up'' letters.

To get permission to use a subset of those letters for our experiment, it was necessary to ensure that all personal information was removed from them. A medical student of the university hospital was therefore paid to manually anonymize 307 of the letters and forward them to us. The letters were given to us in Microsoft Word XML format.

\section{Use Case Scenario}
The recommender we envision is built into the clinical everyday life of doctors. During the visit of a patient, physicians write a new or modify an existing letter for this patient. Already in this writing phase, we wish to retrieve letters of similar patients and present them on demand. Thereby doctors' decision-making processes can be guided by these similar cases if the doctor deems this necessary. The perceived suitability of the retrieved letters has to be exceptionally high. Doctors' additional time resources are usually very limited. Therefore the recommender system will only be used in practice if almost all recommendations are considered useful.

With the dataset mentioned above, we set out to build this kind of recommender system. To achieve this goal we first clear our dataset from duplicates with the help of basic information retrieval methods. This process is described in the subsequent chapter. We elaborate on methods for hiding and showing specific information from the letters in chapter 3. For this goal, we make extensive use of more sophisticated information retrieval methods and introduce them alongside there. Chapter 4 describes the recommender system itself, the experiment we conducted to assess its quality, and the evaluation of the obtained data. In chapter 5 we describe a related use case in a research setting and shortly assess how useful the developed system can be in practice. Finally, we review shortcomings, conceivable problems, and possibilities for further work in the discussion.

\end{document}