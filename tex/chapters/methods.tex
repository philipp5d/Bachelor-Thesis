% !TEX root = ../Thesis.tex
\documentclass[Thesis.tex]{subfiles}
\begin{document}
\chapter{Methods}
\label{ch:methods}

%\section{Notes on Notation and Naming Conventions}
%
%In this thesis vectors are represented by lowercase boldface characters
%such as $\mathbf{v}$ and matrices by uppercase boldface letters such
%as $\mathbf{M}$. 
%
%In the context of Natural Language Processing in this thesis the words ``document''
%and ``text'' are used interchangeably and refer to texts of variable
%length such as sentences, paragraphs or whole documents. ``term''
%and ``word'' are used interchangeably as well. Although ``term'' in general
%often refers to an entity within the text, that might be
%composed of several ``words'', for example ``New York'', here both will be used to refer to single words.


%\section{Preprocessing}
%
%One very common way of representing text in a computer is to view
%the text as a sequence of individual characters, a string. However,
%for many methods of information retrieval and machine learning, including
%the ones described subsequently, the units of text we would like to
%deal with are words or terms not sequences of characters. The process
%of extracting words or terms from a sequence of characters is known
%as tokenization \citep{Manning2008prepr}. A first naive approach to tokenization might be to
%split the sequence of characters on every whitespace and regard everything
%in between as a word. This approach, however, can easily lead to unexpected
%results. Consider the example string ``john loves mary.''. The naive
%approach yields the words ``john'', ``loves'' and ``mary.''.
%Note how the last word contains the punctuation character ``.''.
%So even for very easy examples the naive approach does not produce
%expected results. Luckily many more sophisticated algorithms are implemented
%in ready-to-use open-source software packages.
%
%A second common step in preprocessing is the removal of so called
%stop words. Stop words are usually the most commonly used words of
%a language, that do not contain much meaning, e.g. words like ``the'',
%``a'', ``just'' or ``some''. While they are necessary for the
%grammatical structure of a language, they do not convey much information
%about the similarity of documents and can be disregarded in further
%analysis steps \citep{Manning2008prepr}.
%
%Another often used preprocessing method is stemming. Stemming refers
%to the process of mapping (inflected) words to their stems, e.g. mapping
%words such as ``am'', ``is'' and ``are'' to the word ``be''.
%The rationale behind using stemming is that for finding e.g. the topic
%of a text it should not matter which form of the word is present in
%a text, but only that a word is present in a text \citep{Manning2008prepr}.
%
%All three preprocessing steps described above are used in this thesis
%except where noted otherwise. The python package NLTK (natural language
%toolkit) provides many specific implementations of these algorithms
%and was used throughout this thesis. For tokenization the ``TreebankWordTokenizer''
%and for stemming the ``SnowballStemmer'' are used. A list of German stopwords is also taken from the package.
%
%
%\section{Bag of Words}
%
%After preprocessing of a text one of the standard approaches for representing
%documents for information retrieval and machine learning algorithms
%is in the bag of words model. In the bag of words model a text is
%represented by the multiset (bag) of its words. That means all word
%order information is disregarded and only the information whether
%(or how often) a word is present in a text is encoded \citep{Manning2008vect}. (Mit der Art wie das Buch hier zitiert wird bin ich noch nicht glücklich. Es sollte eigentlich chapter 2 und 6 sein und nicht 2008 a und b.)
%
%In this model documents are represented as fixed length
%feature vectors, where every feature is a kind of word occurrence
%count. In the simplest approach every feature is the term frequency
%$tf(t,d)$ of term $t$ in a document $d$. Where $tf(t,d)$ is
%the occurrence count of word $t$ in document $d$. To compute feature
%vectors for documents in a corpus the vocabulary $V$ of the corpus
%needs to be established first. Documents are represented by a $1\times|V|$
%vector of the values $tf(t,d)$ for all $t\in V$. Assume the corpus
%consists of two documents $d_{1}=$``john loves mary'' and $d_{2}=$``mary
%loves pizza''. The vectors representing the two texts are:
%
%\[
%\textbf{v}_{d_1}=\begin{array}{c}
%1\\
%1\\
%1\\
%0
%\end{array}\ \textbf{v}_{d_2}=\begin{array}{c}
%0\\
%1\\
%1\\
%1
%\end{array}\ \begin{array}{c}
%john\\
%loves\\
%mary\\
%pizza
%\end{array}
%\]
%
%
%This representation obviously has severe limitation, consider for example
%the two sentences $d_{3}=$``john loves mary'' and $d_{4}$=``mary
%loves jon''. Their representation in the bag of words model is equivalent,
%although they express quite different meanings. Still the bag of words
%model is a standard approach for information retrieval problems, as
%it has many desirable properties. For example documents can be expressed
%as a fixed length vector, necessary for many machine learning tasks,
%and distances in the vector space can straight forwardly be interpreted
%as a measure for document dissimilarities. In information retrieval often
%the focus lies on the topic of a text rather than it's specific content.
%Some of the weaknesses of the bag of words approach are then neglectable.
%Consider again the sentences $d_{3}$ and $d_{4}$ and additionally
%the sentence $d_{5}=$''computers automate tasks''. If we are interested
%in the topic of the sentences or in their relevance to each other,
%then the bag of words model will give the reasonable result that $d_{3}$
%and $d_{4}$ are more relevant to each other than to $d_{5}$.


\section{N-Grams}

In order to extend the bag of words representation to capture word
order to some extent n-grams have been proposed. The idea of n-grams
is to treat a sequence of n words as a distinct token or term. For
the sentence $d_{3}$ the list of 2-grams or bigrams is {[}(john,
loves), (loves mary){]} and for $d_{4}$it is {[}(mary, loves), (loves,
john){]}. So the problem of the equivalent representation of $d_{3}$
and $d_{4}$ has been solved by capturing word order information in
small context windows. However, this comes at the cost of a higher
dimensionality of the resulting feature vector as there are more distinct
n-grams in a corpus than single words. In principle the cardinality
of the n-gram vocabulary could be as large as $|V_{n}|=|V_{1}|^{n}$,
where $V_{n}$ is the n-gram vocabulary. However, in natural texts
not all possible n-grams occur and the feature vector dimensionality
grows more moderately than an exponential function with increasing
$n$.   (Vielleicht kommt der Absatz doch wieder raus. Habe n-grams jetzt doch noch gar nicht verwendet.)



\end{document}