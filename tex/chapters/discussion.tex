% !TEX root = ../Thesis.tex
\begin{document}
\documentclass[Thesis.tex]{subfiles}
\chapter{Discussion}
\label{ch:discussion}

%In this thesis, we have developed a prototype for an automatized recommendation system on a physician letter database. We conducted an experiment to evaluate whether or not this system is usable in practice. Thereby we found strong evidence that, indeed, the system performs well enough to deploy it in a clinic.

\section{Theoretical Reflections}
In the motivation section of our introduction, we discussed several cognitive science results about human reasoning. More specifically, we presented evidence for the hypothesis that humans often reason from examples stored in memory. The theory on this aspect of cognitive science is not as clear-cut as we presented it, though. There is, in fact, a great controversy between two different views on this subject. So-called exemplar theories indeed argue that humans reason based on specific examples stored in memory. The contrary viewpoint---prototype theory---suggests that humans form abstract prototypes and cognitive processes are based on these prototypes rather than specific examples \citep{Rosch1975,Bordage1984}.

A second, complicating issue arises, because there is, additionally, evidence that humans reason with conscious hypothesis testing rather than automatic pattern matching (based on examples or prototypes) \citep{Elstein1978}. The latter controversy is resolved by findings that suggest that humans use a flexible approach to problem-solving preferring either pattern matching or hypothesis testing based on the perceived difficulty of the problem. Difficult problems usually cannot be solved by pattern matching and subjects fall back to hypothesis testing \citep{Elstein2002}. Additionally, the pattern matching kind of problem-solving does not occur when presenting information as written feature lists rather than pictorially \citep{Allen1991}. Hence, it is plausible that in our recommender system's use case of physician letters of uncommon cases, doctors apply conscious hypothesis testing. Consequently, the controversy between exemplar-based and prototype-based reasoning is not important for theoretical reflections on the usefulness of our system.

It is highly plausible that in our use case the model of \citet{Klein2008} is applicable. As already described in the introduction, his view of real-world decision-making assumes that subjects generate hypotheses for the current problem based on exemplars stored in memory. The solutions from the exemplars are applied to the current problem and refined or discarded if necessary. In our application scenario, the exemplars are provided by our algorithm rather than retrieved from memory. We can thereby extend the physician's exemplar retrieval to the whole database of the hospital.

\section{Further Research}
As explained above, there is good theoretical ground to assume the usefulness of presenting examples in aiding reasoning about difficult problems. Furthermore, we showed in chapter \ref{ch:recommender_system} that our system can indeed retrieve useful instances. Nevertheless, it remains unclear whether physicians' decisions will improve when using the presented system. Further research can address this issue in the following way: Medical students or unexperienced doctors can be asked to make therapy decisions with and without the recommender system. An expert can then review the benefits in therapy decisions when using our algorithm. While this experiment addresses the quality of therapy decisions as judged by human experts, it still does not address whether patients' health outcomes indeed improve. Only a clinical trial can answer this question.

\section{Future Work}
The next steps for this project consist in extending this prototype to work on the large database of the University Hospital Freiburg. We already foresee several challenges that will arise when working on this extended database. First, new data formats (other than Microsoft Word XML files) will have to be processed. Second, the current algorithms are not optimized for speed or low memory footprint. Surely, these issues must be addressed when extending the database by several orders of magnitude.
Moreover a usable prototype will have to be made available for doctors in the hospital. Initially, we will have to encourage them to use the recommender. After a testing period their evaluation and criticism can be collected and assessed. If yielding positive results, the system can be incorporated into the clinic's documentation system. At that point, recommendations can be made automatically available during the writing of a new letter.
In the long run, it may also be possible to extend the system to work on databases of several clinics. If these databases are used collectively for the recommendation procedure, issues of anonymization will arise. Doctors from one clinic are not allowed to view the patient data from other clinics without the patient's consent. In a realistic scenario, the system might only tell doctors that letters of similar patients are available at other clinics and the doctor has to obtain those letters manually. In this case, though, the usefulness will suffer greatly and it is likely that doctors only draw on this procedure for very uncommon cases, due to the increased effort of obtaining the letters manually.

Still, we believe that our system can improve the health outcomes of patients and should therefore be implemented in a concrete application. From personal feedback of staff of the hospital, it seems indeed likely that the system will be extended to work within the University Hospital Freiburg soon.

\end{document}