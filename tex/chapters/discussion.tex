% !TEX root = ../Thesis.tex
\begin{document}
\documentclass[Thesis.tex]{subfiles}
\chapter{Discussion}
\label{ch:discussion}

In the motivation section of our introduction we discuss several cognitive science results about human reasoning. We present evidence for the assumption that humans often reason from examples stored in memory. The theory on this aspect of cognitive science is not as clear-cut, though. There is, in fact, a great controversy between two different views on this subject. So called exemplar theories indeed argue that humans reason based on specific examples stored in memory. The contrary viewpoint---prototype theory---suggests that humans form abstract prototypes and cognitive processes are based on these prototypes rather than specific examples. A second, complicating issue arises, because there is, additionally, evidence that humans reason with conscious hypothesis testing rather than pattern matching (based on examples or prototypes). In certain situations, it is indeed plausible

In this thesis, we have developed a prototype for an automatized recommendation system on a physician letter database. We conducted an experiment to evaluate whether or not this system is usable in practice. Thereby we found strong evidence that, indeed, the system performs well enough to deploy it in a clinic. The next steps for this project, therefore, consist in extending this prototype to work on the large database of the university hospital Freiburg. We already foresee several challenges that will arise when working on this extended database. First, new data formats (other than Microsoft Word XML files) will have to be processed. Second, the current algorithms are not optimized for speed or low memory footprint. Surely some work will have to address these kinds of issues, when extending the database by several orders of magnitude.

Then a usable prototype will have to be made available for doctors in the hospital. Initially, we will have to encourage them to use the recommender. After a testing period their personal critiques can be collected and assessed. If this yields positive results, the system can be incorporated into the clinic's documentation system. At that point, recommendations can be made automatically available during the writing of a new letter.

It will also be possible to extend the system to work on databases of several clinics. If these databases are used collectively for the recommendation procedure issues of anonymization will arise. Doctors from one clinic are not allowed to view the patient data from other clinics without the patient's consent. In a realistic scenario, the system might only tell doctors, that letters of similar patients are available at other clinics and the doctor has to obtain those letters manually. Thereby the usefulness will suffer greatly, though, and this procedure will surely only be done for very uncommon patients.



\end{document}